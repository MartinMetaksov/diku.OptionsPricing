\chapter{Background}
\label{chapter:background}

\section{Financial background}
Derivatives have become important in finance in the last 30 years with futures and options actively traded worldwide. The derivatives market is much bigger than the stock market when underlying assets are measured. Derivatives can be used for hedging, speculation or arbitrage as they transfer a wide range of risks in the economy from one entity to another. A \textit{derivative} is a financial instrument with a value that derives from the value of some other basic underlying asset.~\cite[pg.1]{ofod}

\subsection{Options}
This thesis will focus on one type of derivatives - \textit{options}. These are contracts that give the holder the right to buy or sell an underlying asset at a certain point in time for a certain price, both specified when purchasing the option. This is in contrast with other derivatives - \textit{forwards} and \textit{futures}, where the holder is obligated to buy or sell the underlying asset.
\paragraph{}
There are two types of options. A \textit{call option} gives the holder the right to buy the underlying asset by a certain date for a certain price. A \textit{put option} gives the holder the right to sell the underlying asset by a certain date for a certain price. The price in the contract is called \textit{strike price} and the expiration date is called \textit{maturity}. Options that can be exercised at any time before maturity are known as \textit{American options} and options that can be exercised only on the expiration date itself are known as \textit{European options}. One contract usually allows to buy or sell 100 shares.~\cite[pg.7-8]{ofod}
\paragraph{Example}
An investor spent 20.000 kr for an option to buy 100 Maersk shares for 9.600 kr each. The current market price for Maersk stock is 9.440 kr as of March 15, 2018. If the price does not rise above 9.600 kr by the maturity, the investor does not exercise the option and loses 20.000 kr. However, if Maersk stock is priced at 10.000 kr when the option can be exercised, the investor is able to buy 100 shares for the strike price of 9.600 kr and immediately sell them for 10.000 kr. This will generate a profit of $400 * 100 = 40.000$ kr minus the initial contract cost of 20.000 kr.

Table~\ref{table:option-sell} illustrates an example of exercising this option at different dates. Even if the stock price rises above the strike price, the net profit might still be negative when the contract price is accounted for.  

\begin{table}[h]
\centering
\caption{Profit generated by a call option with strike price of 9.600 kr and contract price of $100 \times 200 = 20.000$ kr.}
\label{table:option-sell}
\begin{tabular}{|l|l|l|l|l|}
\hline
                       & March 2018    & June 2018 & Sept. 2018 & Dec. 2018 \\ \hline
Stock price (kr)       & 9.440         & 9.700     & 10.000         & 9.800         \\ \hline
Share sale profit (kr) & not exercised & 10.000    & 40.000         & 20.000        \\ \hline
Net profit (kr)        & -20.000       & -10.000   & 20.000         & 0             \\ \hline
\end{tabular}
\end{table}

\subsection{Bonds}
Bonds are a form of debt that allow companies or governments borrow money from investors. Most bonds periodically paid coupons (interest) to the holder. The principal payment (known as face value or par value) is paid at the end of bond's life.~\cite[pg.80]{ofod}

\paragraph{Zero rates}
Interest rate on an investment that starts today and lasts for $n$ years are called $n$-year zero-coupon interest rates. All the interest and principal is realized at the end. For example, an investment of 100\$ with a 5-year zero rate with continuous compounding at 5\% p.a. grows to~\cite[pg.80]{ofod}
\begin{equation*}
    100 \times e^{0.05 \times 5} = 128.40
\end{equation*}

\paragraph{Bond pricing}
The theoretical price of a bond can be calculated as the present value of all cash flows received by the holder. For example, there is a 2-year bond with 100\$ principal that pays coupons at the rate of 6\% semiannually. When the zero rates are as in table~\ref{table:zero-rates}, we can calculate the present value of the first coupon of 3\$ by discounting it at 5.0\% for 6 months, second coupon at 1 year and so on. Then the theoretical price of the bond is~\cite[pg.80-81]{ofod}
\begin{equation*}
    3e^{-0.05 \times 0.5} + 3e^{-0.058 \times 1.0} + 3e^{-0.064 \times 1.5} + 103e^{-0.068 \times 2.0} = 98.39
\end{equation*}

\begin{table}[h]
\centering
\caption{Zero rates used for bond pricing}
\label{table:zero-rates}
\begin{tabular}{|l|l|}
\hline
Maturity (years) & \begin{tabular}[c]{@{}l@{}}Zero rate (\%)\\ (continuously compounded)\end{tabular} \\ \hline
0.5              & 5.0                                                                                \\ \hline
1.0              & 5.8                                                                                \\ \hline
1.5              & 6.4                                                                                \\ \hline
2.0              & 6.8                                                                                \\ \hline
\end{tabular}
\end{table}

\paragraph{Bond yield}
A bond's yield is the single discount rate that gives a bond price equal to its market price. Continuing the pricing example, the yield $y$ can be computed as
\begin{equation*}
    3e^{-y \times 0.5} + 3e^{-y \times 1.0} + 3e^{-y \times 1.5} + 103e^{-y \times 2.0} = 98.39
\end{equation*}
This equation gets solved by setting $y = 6.76\%$.~\cite[pg.81]{ofod}


\subsection{Trinomial trees}
- what are they best for and why?
* analytical tractability
* path-dependent derivatives -> exotics
* small-dimensions derivatives -> dependent on two/three stochastic variables at most, one-factor, two-factor 
\subsection{Other numerical methods}
- finite difference methods (FDM) to solve partial differential equations (POE1)
- Monte Carlo simulations
\subsection{Yield curve}
\subsection{Volatility curve}

\section{Flattening background}
...